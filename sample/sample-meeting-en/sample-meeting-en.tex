\documentclass[lualatex,meeting,english]{improceedings}

% Packages

% Macros defined by the author
\newcommand{\pkg}[1]{\textsf{#1}}
\newcommand{\code}[1]{\texttt{#1}}

\title{\pkg{IMProceedings} Document Class Sample Document for Meeting}
\date{2024/06/01}
\group{Learning Group}
\grade{M1}
\author{Yuichiro Iwashita}

\begin{document}

\maketitle

\section{Introduction}

\pkg{IMProceedings} document class was created for writing internal presentation manuscripts at the Intelligent Media Processing Research Group.
In addition, options for meeting materials are also available.
Please note that the \pkg{IMProceedings} document class is not compatible with the \pkg{imp} style file.

\subsection{Document Style}

You can specify the document style using the document class options.
\begin{verbatim}
  \documentclass[lualatex,STYLE]{improceedings}
\end{verbatim}
\begin{itemize}
  \item \code{presentation}: For research presentation materials (default)
  \item \code{meeting}: For meeting materials
\end{itemize}
Also, you can specify the language using the \code{english} option.

\section{Fonts}

In the \pkg{IMProceedings} document class, the fonts used are as follows:
\begin{itemize}
  \item Japanese: Haranoaji fonts
  \item English and math: Times series fonts (\pkg{newtxtext}, \pkg{newtxmath})
  \item Calligraphy: \pkg{mathalfa} package
\end{itemize}

\section{Header}

\pkg{IMProceedings} document class displays the author's name, research group, grade, type of material, presentation date, and presentation title in the header of the first page.
These are specified by the \code{author}, \code{group}, \code{grade}, \code{term}, \code{date}, and \code{title} commands.
For example, the header information of this document is as follows.

\begin{verbatim}
  \title{\pkg{IMProceedings} Document ...}
  \term{Lab Presentation Manuscript ...}
  \date{2024/06/01}
  \group{Learning Group}
  \grade{M1}
  \author{Yuichiro Iwashita}
\end{verbatim}

\section{Tables}

To create tables, use the standard \LaTeX environments.
Table~\ref{tab:sample} is a table created using the \code{table} environment and the \code{tabular} environment.

\begin{table}[tbh]
  \centering
  \caption{Sample Table}
  \label{tab:sample}
  \begin{tabular}{c|c|c}
    \hline
    A & B & C \\
    \hline
    1 & 2 & 3 \\
    4 & 5 & 6 \\
    \hline
  \end{tabular}
\end{table}

You can also use the \code{table*} environment to place a table that spans the entire width of the page.
Table~\ref{tab:sample2} is a table created using the \code{table*} environment.

\begin{table*}[tbh]
  \centering
  \caption{Sample Table 2}
  \label{tab:sample2}
  \begin{tabular}{c|c|c}
    \hline
    A & B & C \\
    \hline
    1 & 2 & 3 \\
    4 & 5 & 6 \\
    \hline
  \end{tabular}
\end{table*}

\section{Figures}

To insert figures, use the \code{figure} environment and the \code{includegraphics} command.
Figure~\ref{fig:sample} is a figure inserted using the \code{figure} environment and the \code{includegraphics} command.

\begin{figure}[tbh]
  \centering
  \includegraphics[width=0.5\columnwidth]{example-image-a}
  \caption{Sample Figure}
  \label{fig:sample}
\end{figure}

As with tables, you can use the \code{figure*} environment to place a figure that spans the entire width of the page.

\begin{figure*}[tbh]
  \centering
  \includegraphics[width=0.5\linewidth]{example-image-b}
  \caption{Sample Figure 2}
  \label{fig:sample2}
\end{figure*}

\section{Equations}

There are three ways to display equations: inline equations, numbered display equations, and unnumbered display equations.
Inline equations are enclosed in \code{\$...\$}.
For example, $Y = XW + b$ is written as \code{\$Y = XW + b\$}.
Numbered display equations are written using the \code{equation} environment.

\begin{equation}
  Y = XW + b
\end{equation}

Unnumbered display equations are written using the \code{equation*} environment or by enclosing the equation in \code{\textbackslash [...\textbackslash ]}.

\begin{equation*}
  Y = XW + b
\end{equation*}

To display equations in bold ($\boldsymbol{a}$), use \code{\textbackslash boldsymbol\{...\}}.
To display equations in non-italic form ($\mathbf{a}$), use \code{\textbackslash mathbf\{...\}}.
To display equations in blackboard bold ($\mathbb{R}$), use \code{\textbackslash mathbb\{...\}}.
To display equations in script ($\mathscr{L}$), use \code{\textbackslash mathscr\{...\}}.
To display equations in calligraphy ($\mathcal{L}$), use \code{\textbackslash mathcal\{...\}}.

\section{Citations}

It is recommended to use \pkg{BibTeX} for citing references.
In this sample document, the following code is written just before \code{\textbackslash end\{document\}} to output the list of references.

\begin{verbatim}
  \bibliographystyle{unsrt}
  \bibliography{sample-base}
\end{verbatim}

The default bibliography style is \code{unsrt}, but you can change it as needed.

To cite a reference, use \code{\textbackslash cite\{...\}}.
For example, Yamada et al. proposed ShakeDrop~\cite{yamada2019shakedrop}.
Ishimaru et al. proposed an activity recognition method using Google Glass~\cite{ishimaru2014blink}.

\section{Pseudocode}

To write pseudocode, use the \pkg{algorithm} and \pkg{algorithmic} environments.
Algorithm~\ref{alg:sample} is a sample pseudocode.

\begin{algorithm}
  \caption{Compute $aˆb$}
  \begin{algorithmic}[1]
    \State{$c \gets 1$}
    \While{$b \geq 0$}
    \State{$c \gets ac$}
    \State{$b \gets b-1$}
    \EndWhile
  \end{algorithmic}
  \label{alg:sample}
\end{algorithm}

\section{Miscellaneous}

To insert footnotes, use \code{\textbackslash footnote\{...\}}.
For example, Google~\footnote{\url{http://www.google.co.jp}} is written as shown.
To insert a URL, use \code{\textbackslash url\{...\}}.

\section*{Acknowledgments}

This document class was created based on the \pkg{jlreq} and \pkg{NLProceedings} document classes.

\section*{Last Updated}

\date{\today}

% References
\bibliographystyle{unsrt}
\bibliography{sample-base}

\end{document}